%%
% Plantilla de Memoria
% Modificación de una plantilla de Latex de Nicolas Diaz para adaptarla 
% al castellano y a las necesidades de escribir informática y matemáticas.
%
% Editada por: Mario Román
%
% License:
% CC BY-NC-SA 3.0 (http://creativecommons.org/licenses/by-nc-sa/3.0/)
%%

%%%%%%%%%%%%%%%%%%%%%
% Thin Sectioned Essay
% LaTeX Template
% Version 1.0 (3/8/13)
%
% This template has been downloaded from:
% http://www.LaTeXTemplates.com
%
% Original Author:
% Nicolas Diaz (nsdiaz@uc.cl) with extensive modifications by:
% Vel (vel@latextemplates.com)
%
% License:
% CC BY-NC-SA 3.0 (http://creativecommons.org/licenses/by-nc-sa/3.0/)
%
%%%%%%%%%%%%%%%%%%%%%

%----------------------------------------------------------------------------------------
%	PAQUETES Y CONFIGURACIÓN DEL DOCUMENTO
%----------------------------------------------------------------------------------------

%% Configuración del papel.
% microtype: Tipografía.
% mathpazo: Usa la fuente Palatino.
\documentclass[a4paper, 11pt]{article}
\usepackage[protrusion=true,expansion=true]{microtype}
\usepackage{mathpazo}

% Indentación de párrafos para Palatino
\setlength{\parindent}{0pt}
  \parskip=8pt
\linespread{1.05} % Change line spacing here, Palatino benefits from a slight increase by default


%% Castellano.
% noquoting: Permite uso de comillas no españolas.
% lcroman: Permite la enumeración con numerales romanos en minúscula.
% fontenc: Usa la fuente completa para que pueda copiarse correctamente del pdf.
\usepackage[spanish,es-noquoting,es-lcroman]{babel}
\usepackage[utf8]{inputenc}
\usepackage[T1]{fontenc}
\selectlanguage{spanish}

\makeatletter

%----------------------------------------------------------------------------------------
%	TÍTULO
%----------------------------------------------------------------------------------------
% Configuraciones para el título.
% El título no debe editarse aquí.
\renewcommand{\maketitle}{
  \begin{flushright} % Right align
  
  {\LARGE\@title} % Increase the font size of the title
  
  \vspace{50pt} % Some vertical space between the title and author name
  
  {\large\@author} % Author name
  \\\@date % Date
  \vspace{40pt} % Some vertical space between the author block and abstract
  \end{flushright}
}

% Título
\title{\textbf{How to disable Bomba\_Nucelar}\\ % Title
Práctica 3 de EC} % Subtitle

\author{\textsc{Óscar Bermúdez Garrido} % Author
\\{\textit{Universidad de Granada}}} % Institution

\date{\today} % Date



%----------------------------------------------------------------------------------------
%	DOCUMENTO
%----------------------------------------------------------------------------------------

\begin{document}

\maketitle % Print the title section

%% Inicio del documento

	\section{Mecanismo interno}
		La bomba pide 2 contraseñas, una alfanumérica y otra sólo numérica y realiza una serie de comparaciones
		con varias contraseñas de cada uno de los 2 tipos que dispone.
		
		Estas comparaciones están pensadas para que el resultado de las comparaciones si las contraseñas
		introducidas son correctas permita desactivarla pero si alguna es incorrecta, el resultado será su
		detonación.
		
		La bomba en sí, está implementada para tener código innecesario cuyo único propósito es complicar el
		proceso de desensamblado, por ende, complica la desactivación y asegurándose que para lograr desactivarla
		se ha comprendido perfectamente el funcionamiento de la misma. Pues si nos limitamos a tomar los primeros
		valores que se comparan, acabará explotando.
		
		Además, en el hipotético caso de que transcurra demasiado tiempo en la introducción de las
		contraseñas(una hora), detonará independientemente de si eran las correctas.

	\section{Desactivación}
		Hacemos un volcado sobre un archivo nuevo, el cuál abriremos con un editor de texto.
		
		A continuación, ejecutaremos nuestro desensamblador(en mi caso, \textit{ddd}) sobre \textbf{bomba\_nucelar},
		
		Buscamos la función \textit{main} y la copiamos en un nuevo archivo. Sobre esta función, buscamos cada
		vez que utilizamos \textit{cmp} y \textit{strncmp}:
		
		\begin{itemize}
			\item Cada vez que aparece \textit{cmp}, estamos comparando números, por tanto, tenemos que pedir a
			nuestro desensamblador que nos muestre el contenido del registro como decimal, los posibles valores
			que podemos ver serán el tiempo que permito como máximo para desactivar la bomba y las candidatas a
			claves numéricas. Abriremos otro archivo e introduciremos éstas.

			\item Cada vez que aparece \textit{strncmp}, estamos comparando cadenas de caracteres, por tanto,
			tenemos que mirar un par de instrucciones antes y veremos la dirección desde la que se está introduciendo
			y pediremos a nuestro desensamblador que nos muestro el contenido de dicha dirección como texto para
			así poder ver la clave alfanumérica. Introduciremos éstas en el archivo anteriormente abierto.
		\end{itemize}

		Luego, con todas las combinaciones de contraseñas que hemos sacado, podemos intentar introducir cada 
		combinación posible ya que son pocas o podemos intentar comprender el código paso a paso, lo cuál
		puede parecer imposible pero es más sencillo de lo que parece(además de aportar un interesante ejercicio
		para ver si se comprendió correctamente el lenguaje ensamblador) y ver qué valores hacen detonar la
		bomba, por tanto, qué comparaciones son necesarias cumplir y cuáles no, y por tanto, saber qué contraseñas
		son las correctas.

\end{document}
